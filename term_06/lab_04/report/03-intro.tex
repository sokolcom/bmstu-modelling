
\section*{Цель работы}
Получение навыков разработки алгоритмов решения смешанной краевой задачи при реализации моделей, построенных на квазилинейном уравнении параболического типа.






\section*{Постановка задачи}
Задана математическая модель. Уравнение для функции $T(x, t)$.
\begin{equation}
c(T)\frac{\partial T}{\partial t} = \frac{\partial}{\partial x}(k(T)\frac{\partial T}{\partial x}) - \frac{2}{R}\alpha(x)T + \frac{2T_0}{R}\alpha(x)
\end{equation}

Краевые условия:
$$
\left\{\begin{matrix}
	t = 0, T(x, 0) = T_0, \\ 
	x = 0, -k(T(0))\frac{\partial T}{\partial x} = F_0,\\
	x = l, 	-k(T(l))\frac{\partial T}{\partial x} = \alpha_N(T(l) - T_0)
\end{matrix}\right
$$

В обозначениях уравнения лекции:
$$
p(x) = \frac{2}{R}\alpha(x), f(u) = f(x) = \frac{2T_0}{R}\alpha(x).
$$


Разностная схема с разностным краевым условием при $x = 0$:
\begin{equation}
	\hat{A}_n\hat{y}_{n - 1} - \hat{B_n}\hat{y}_n + \hat{D}\hat{y}_{n + 1} = - \hat{F}_n
\end{equation}
 
\begin{equation}
	\begin{array}{1}
	(\frac{h}{8}\hat{c}_{\frac{1}{2}} + \frac{h}{4}\hat{c}_0 + \hat{\chi}_{\frac{1}{2}} \frac{\tau}{h} + \frac{\tau h}{8}p_{\frac{1}{2}} + \frac{\tau h}{4}p_0)\hat{y}_0 + 
	(\frac{h}{8}\hat{c}_{\frac{1}{2}} - \hat{\chi}_{\frac{1}{2}} \frac{\tau}{h} + \frac{\tau h}{8}p_{\frac{1}{2}})\hat{y}_1 = \\
	 = \frac{h}{8}\hat{c}_{\frac{1}{2}}(y_0 + y_1) + \frac{h}{4}\hat{c}_0 y_0 + \hat{F}\tau + \frac{\tau h}{4}(\hat{f}_{\frac{1}{2}} + \hat{f}_0)
 	\end{array}
\end{equation} 
 
Разностная схема с разностным краевым условием при $x = l$:
 
 \begin{equation}
	 \begin{array}{1}
 	\int_{x_{N - \frac{1}{2}}}^{x_N}dx \int_{t_m}^{t_{m + 1}} c(T)\frac{\partial T}{\partial t}dt = \\
 	\int_{x_{N - \frac{1}{2}}}^{x_N}dx \int_{t_m}^{t_{m + 1}} f(x)dt - \int_{t_m}^{t_{m + 1}}dt \int_{x_{N - \frac{1}{2}}}^{x_N} \frac{\partial F}{\partial x}dx - \int_{x_{N - \frac{1}{2}}}^{x_N}dx \int_{t_m}^{t_{m + 1}} p(x)Tdt
	\end{array}
\end{equation} 

Применим метод правых прямоугольников для интегралов из правой части:

\begin{equation}
	\begin{array}{1}
		\int_{x_{N - \frac{1}{2}}}^{x_N} \hat{c}(\hat{T} - T) dx = \int_{x_{N - \frac{1}{2}}}^{x_N} \hat{f} \tau dx - \int_{t_m}^{t_{m + 1}} (F_N - F_{N - \frac{1}{2}})dt - \int_{x_{N - \frac{1}{2}}}^{x_N} p\hat{T}\tau dx	
	\end{array}
\end{equation} 

Интеграл $ \int_{t_m}^{t_{m + 1}} (F_N - F_{N - \frac{1}{2}})dt $  решим с помощью метода правых прямоугольников, а оставшиеся методом трапеций:

\begin{equation}
	\begin{array}{1}
	\frac{h}{4}[\hat{c}_N (\hat{y}_N - y_N) + \hat{c}_{N - \frac{1}{2}} (\hat{y}_{N - \frac{1}{2}} - y_{N - \frac{1}{2}})] = \\
	= \frac{h}{4}\tau(\hat{f}_N - \hat{f}_{N - \frac{1}{2}}) - \tau (\hat{F}_N - \hat{F}_{N - \frac{1}{2}}) - \frac{h}{4}\tau (p_N \hat{y}_N + p_{N - \frac{1}{2}} \hat{y}_{N - \frac{1}{2}})
	\end{array}
\end{equation} 
 
 
Подставим в выражения для потока:

\begin{equation}
	\begin{array}{1}
	\frac{h}{4}[\hat{c}_N (\hat{y}_N - y_N) + \hat{c}_{N - \frac{1}{2}} (\frac{\hat{y}_N + \hat{y}_{N - 1}}{2} - \frac{y_N + y_{N - 1}}{2})] = \\
	= \frac{h}{4}\tau(\hat{f}_N - \hat{f}_{N - \frac{1}{2}}) - \tau (\alpha_N(\hat{y}_N - T_0) - \hat{\chi}_{N - \frac{1}{2}} \frac{\hat{y}_{N - 1} - \hat{y}_N}{h}) - \\
	- \frac{h}{4}\tau (p_N \hat{y}_N + p_{N - \frac{1}{2}} \frac{\hat{y}_N + \hat{y}_{N - 1}}{2})	
	\end{array}
\end{equation} 
 
 
Приведем к общему виду:
 
\begin{equation}
	\begin{array}{1}
		(\frac{h}{4} \hat{c}_N + \frac{h}{8} \hat{c}_{N - \frac{1}{2}} + \alpha_N \tau + \frac{\tau}{h}\chi_{N - \frac{1}{2}} + \frac{h}{4}\tau p_N + \frac{h}{8}\tau p_{N - \frac{1}{2}})y_N + \\
		+ (\frac{h}{8} \hat{c}_{N - \frac{1}{2}} - \frac{\tau}{h}\chi_{N - \frac{1}{2}} + \frac{h}{8}\tau p_{N - \frac{1}{2}})y_{N - 1} = \\
		= \frac{h}{4}\hat{c}_N y_N + \frac{h}{8}\hat{c}_{N - \frac{1}{2}}(y_N + y_{N - 1}) + \tau \alpha_N T_0 + \frac{h}{4}\tau (\hat{f}_N + \hat{f}_{N - \frac{1}{2}})
	\end{array}
\end{equation} 
 

Применим простую аппроксимацию:

$$
p_{N - \frac{1}{2}} = \frac{p_N + p_{N - 1}}{2}, \hat{f}_{N - \frac{1}{2}} = \frac{\hat{f}_N + \hat{f}_{N - 1}}{2}, \hat{c}_{N - \frac{1}{2}} = \frac{\hat{c}_N + \hat{c}_{N - 1}}{2}
$$
 
 
Если $c(u) = 0$ и сократить $\tau$ формула (8) перейдёт формулу для разностного краевого условия при $x = l$ из предыдущей лабораторной работы.
 

Значения параметров для отладки (все размерности согласованы):
$$ k(T) = a_1(b_1 + c_1 T^{m_1}), $$
$$ c(T) = a_2 + b_2 T^{m_2} - \frac{c_2}{T^2}, $$
$$ a_1 = 0.0134, b_1 = 1, c_1 = 4.35 * 10^{-4}, m_1 = 1, $$
$$ a_2  = 2.049, b_2 = 0.563 * 10^{-3}, c_2 = 0.528 * 10^5, m_2 = 1, $$
$$ \alpha(x) = \frac{c}{x - d}, \alpha_0 = 0.05, \alpha_N = 0.01, $$
$$ l = 10, T_0 = 300, R = 5, $$
$$ F(t) = 50.$$





\section*{Листинг кода}

\lstset{ %
	language=python,                
	basicstyle=\small\sffamily, % размер и начертание шрифта для подсветки кода
	numbers=left,               % где поставить нумерацию строк (слева\справа)
	numberstyle=\tiny,           % размер шрифта для номеров строк
	stepnumber=1,                   % размер шага между двумя номерами строк
	numbersep=5pt,                % как далеко отстоят номера строк от подсвечиваемого кода
	showspaces=false,            % показывать или нет пробелы специальными отступами
	showstringspaces=false,      % показывать или нет пробелы в строках
	showtabs=false,             % показывать или нет табуляцию в строках
	frame=single,              % рисовать рамку вокруг кода
	tabsize=4,                 % размер табуляции по умолчанию равен 2 пробелам
	captionpos=b,              % позиция заголовка вверху [t] или внизу [b] 
	breaklines=true,           % автоматически переносить строки (да\нет)
	breakatwhitespace=false, % переносить строки только если есть пробел
	escapeinside={\#*}{*)}   % если нужно добавить комментарии в коде
}

На листинге 1 приведен код программы на языке Python 3.

\begin{lstlisting}[caption=Код программы]
import matplotlib.pyplot as plt
import numpy as np
from math import fabs

a1 = 0.0134
b1 = 1
c1 = 4.35e-4
m1 = 1
a2 = 2.049
b2 = 0.563e-3
c2 = 0.528e5
m2 = 1
alpha0 = 0.05
alphaN = 0.01
l = 10
T0 = 300
R = 0.5
F0 = 50
h = 1e-3
t = 1

def k(T):
	return a1 * (b1 + c1 * T ** m1)


def c(T):
	return a2 + b2 * T ** m2 - (c2 / T ** 2)


def alpha(x):
	d = (alphaN * l) / (alphaN - alpha0)
	c = - alpha0 * d
	return c / (x - d)


def p(x):
	return 2 * alpha(x) / R


def f(x):
	return 2 * alpha(x) * T0 / R


def func_plus_half(x, step, func):
	return (func(x) + func(x + step)) / 2


def func_minus_half(x, step, func):
	return (func(x) + func(x - step)) / 2


def A(T):
	return t / h * func_minus_half(T, t, k)


def D(T):
	return t / h * func_plus_half(T, t, k)


def B(x, T):
	return A(T) + D(T) + c(T) * h + p(x) * h * t


def F(x, T):
	return f(x) * h * t + c(T) * T * h


def left_boundary_condition(T_prev):
	K0 = h / 8 * func_plus_half(T_prev[0], t, c) + h / 4 * c(T_prev[0]) + \
		func_plus_half(T_prev[0], t, k) * t / h + \
		t * h / 8 * p(h / 2) + t * h / 4 * p(0)
	M0 = h / 8 * func_plus_half(T_prev[0], t, c) - \
		func_plus_half(T_prev[0], t, k) * t / h + \
		t * h * p(h / 2) / 8
	P0 = h / 8 * func_plus_half(T_prev[0], t, c) * (T_prev[0] + T_prev[1]) + \
		h / 4 * c(T_prev[0]) * T_prev[0] + \
		F0 * t + t * h / 8 * (3 * f(0) + f(h))
	return K0, M0, P0


def right_boundary_condition(T_prev):
	KN = h / 8 * func_minus_half(T_prev[-1], t, c) + h / 4 * c(T_prev[-1]) + \
		func_minus_half(T_prev[-1], t, k) * t / h + t * alphaN + \
		t * h / 8 * p(l - h / 2) + t * h / 4 * p(l)
	MN = h / 8 * func_minus_half(T_prev[-1], t, c) - \
		func_minus_half(T_prev[-1], t, k) * t / h + \
		t * h * p(l - h / 2) / 8
	PN = h / 8 * func_minus_half(T_prev[-1], t, c) * (T_prev[-1] + T_prev[-2]) + \
		h / 4 * c(T_prev[-1]) * T_prev[-1] + t * alphaN * T0 + \
		t * h / 4 * (f(l) + f(l - h / 2))
	return KN, MN, PN


def run(T_prev, K0, M0, P0, KN, MN, PN):
	eps = [0, -M0 / K0]
	eta = [0, P0 / K0]
	x = h
	n = 1

	while (x + h < l):
		eps.append(D(T_prev[n]) / (B(x, T_prev[n]) - A(T_prev[n]) * eps[n]))
		eta.append((F(x, T_prev[n]) + A(T_prev[n]) * eta[n]) / (B(x, T_prev[n]) - A(T_prev[n]) * eps[n]))
		n += 1
		x += h

	y = [0] * (n + 1)
	y[n] = (PN - MN * eta[n]) / (KN + MN * eps[n])
	for i in range(n - 1, -1, -1):
		y[i] = eps[i + 1] * y[i + 1] + eta[i + 1]

	return y


def main():
	step1 = int(l / h)
	T = [T0] * (step1 + 1)
	T_new = [0] * (step1 + 1)
	ti = 0
	res = []
	res.append(T)
	
	while True:
		prev = T
		while True:
			K0, M0, P0 = left_boundary_condition(prev)
			KN, MN, PN = right_boundary_condition(prev)
			T_new = run(prev, K0, M0, P0, KN, MN, PN)
			max = fabs((T[0] - T_new[0]) / T_new[0])
	
			for step2, j in zip(T, T_new):
				d = fabs(step2 - j) / j
				if d > max:
					max = d
			if max < 1:
				break
	
			prev = T_new
	
		res.append(T_new)
		ti += t
	
		check_eps = 0
		for i, j in zip(T, T_new):
			if fabs((i - j) / j) > 1e-2:
				check_eps = 1
		if check_eps == 0:
			break
		T = T_new
	
	x = [i for i in np.arange(0, l, h)]
	te = [i for i in range(0, ti, t)]
	step1 = 0
	for i in res:
		if (step1 % 2 == 0):
			plt.plot(x, i[:-1])
		step1 += 1
	
	plt.plot(x, res[-1][:-1])
	plt.xlabel("x, cm")
	plt.ylabel("T, K")
	plt.grid()
	plt.show()
	
	step2 = 0
	while (step2 < l / 3):
		point = [j[int(step2 / h)] for j in res]
		plt.plot(te, point[:-1])
		step2 += 0.1
	
	plt.xlabel("t, sec")
	plt.ylabel("T, K")
	plt.grid()
	plt.show()


if __name__ == "__main__":
	main()
\end{lstlisting}





\section*{Выполнение заданий лабораторной работы}

\subsection*{Представить разностный аналог краевого условия при $x = l$ и его краткий вывод интегро-интерполяционным методом}

Вывод разностный аналог краевого условия при $x = l$ приведен выше (уравнения (4)-(8)).


\subsection*{График зависимости температуры $T(x, t_m)$ от координаты $x$ при нескольких фиксированных значениях времени $t_m$ (аналогично рисунку в лекции) при заданных выше параметрах. Обязательно представить распределение $T(x, t_m)$ в момент  времени, соответствующий установившемуся режиму, когда поле перестает меняться с некоторой точностью т.е.  имеет  место выход на  стационарный режим. На этой стадии левая часть дифференциального уравнения близка к нулю. График зависимости $T(x_n, t)$ при нескольких фиксированных значениях координаты $x_n$. Обязательно представить случай $n = 0$, т.е. $x = x_0 = 0$.}

На рисунке 1 приведены графики зависимости температуры $T(x, t_m)$ от координаты $x$ при нескольких фиксированных значениях времени $t_m$.

\img{110mm}{1}{Графики зависимости $T(x, t_m)$}


На рисунке 2 приведены графики зависимости $T(x_n, t)$ при нескольких фиксированных значениях координаты $x_n$.

\img{110mm}{2}{Графики зависимости $T(x_n, t)$}







\section*{Ответы на контрольные вопросы}

\subsubsection{1. Приведите  результаты тестирования  программы (графики, общие соображения, качественный анализ)}

\begin{enumerate}
	\item При отрицательном тепловом потоке (например, при $F = -5$) слева идет съем тепла. На рисунках 3 и 4 приведены графики зависимости температуры от координаты и от времени, соответственно.
	
	\img{100mm}{3}{Графики зависимости $T(x, t_n)$ при $F = -5$}
	\img{100mm}{4}{Графики зависимости $T(x_n, t)$ при $F = -5$}
	
	\item Если тепловой поток равен нулю ($F = 0$), то температура стрежня будет равняться температуре окружающей среды, $T = 300$ (см. рис. 5).
	
	\img{100mm}{5}{Графики зависимости $T(x_n, t)$ при $F = 0$ (все графики совпадают)}
	
	\item Если после разогрева стержня сделать тепловой поток равным 0, то стержень будет остывать пока температура не выровняется по всей длине и не станет равной температуре окружающей среды (см. рис. 6).
	
	\img{100mm}{6}{Графики зависимости $T(x_n, t)$}
	
	\item При  произвольной  зависимости потока $F(t)$ от  времени  температурное  поле будет как-то сложным образом отслеживать поток (см. рис. 7). На рисунке 8 приведен график зависимости теплового потока от времени (поток линейно изменялся со значения $F = 50$ до $F = -5$ и затем фиксировался на этом значении).
	
	\img{100mm}{7}{Графики зависимости $T(x_n, t)$ при изменяющемся значении $F(t)$}
	\img{100mm}{8}{Графики зависимости $F(t)$}
	
\end{enumerate}











