
\section*{Цель работы}
Целью данной лабораторной работы является получение навыков разработки алгоритмов решения задачи Коши при реализации моделей, построенных на системе ОДУ, с использованием метода Рунге-Кутта 4-го порядка точности.






\section*{Постановка задачи}
Задана система электротехнических уравнений, описывающих разрядный контур, включающий постоянное активное сопротивление $R_k$, нелинейное сопротивление $R_p(I)$, зависящее от тока $I$, индуктивность $L_k$ и ёмкость $C_k$.
$$
\left\{\begin{matrix}
	\frac{dI}{dT} = \frac{U - (R_k + R_p(I))I}{L_k} \\ 
	\frac{dU}{dt} = -\frac{I}{C_k}\\
\end{matrix}\right.
$$

Начальные условия: $t = 0,I = I_0,U = U_0$ Здесь $I, U$ - ток и напряжение на конденсаторе. Сопротивление $R_p$ рассчитать по формуле:
$$
R_p = \frac{l_k}{2\pi R^2 \int_{0}^{1} \sigma (T(z))z\,dz }
$$

Для функции $T(z)$ применить выражение $T_z = T_0+ (T_w - T_0)z^m$. Параметры $T_0$, $m$ находятся интерполяцией из таблицы 1 при известном токе $I$. Коэффициент электропроводности $\sigma(T)$ зависит от $T$ и рассчитывается интерполяцией из таблицы 2.

\newpage



\begin{center}
	Таблица 1. \\
	\begin{tabular}{ |c|c|c| }
		\hline
		I, A & To, K & m \\ \hline
		0.5 & 6730 & 0.50 \\
		1 & 6790 & 0.55 \\
		5 & 7150 & 1.7 \\
		10 & 7270 & 3 \\
		50 & 8010 & 11 \\
		200 & 9185 & 32 \\
		400 & 10010 & 40 \\
		800 & 11140 & 41 \\
		1200 & 12010 & 39 \\
		\hline
	\end{tabular}
\end{center}



\begin{center}
	Таблица 2. \\
	\begin{tabular}{ |c|c| }
		\hline
		T, K & $\sigma$, 1/Ом см \\ \hline
		4000 & 0.031 \\
		5000 & 0.27 \\
		6000 & 2.05 \\
		7000 & 6.06 \\
		8000 & 12.0 \\
		9000 & 19.9 \\
		10000 & 29.6 \\
		11000 & 41.1 \\
		12000 & 54.1 \\
		13000 & 67.7 \\
		14000 & 81.5 \\
		\hline
	\end{tabular}
\end{center}

\begin{center}
	Параметры разрядного контура: \\
	\begin{tabular}{ c }
		$R = 0.35$ см \\
		$l_e = 12$ см \\
		$L_k =1 87 * 10^-6$ Гн \\
		$C_k = 268 * 10^-6$ Ф \\ 
		$R_k = 0.25$ Ом \\
		$U_co = 1400$ В \\
		$I_o = 0..3$ A \\
		$T_w = 2000$ K \\
	\end{tabular}
\end{center}
Для справки: при указанных параметрах длительность импульса около 600 мкс, максимальный ток – около 800 А.









\section*{Метод Рунге-Кутта 4-ого порядка}

Порядок точности метода: $O(h^4)$.
$$y_{n+1} = y_n + \frac{k_1 + 2k_2 + 2k_3 + k_4}{6},$$
где 
$$k_1 = h_n \phi(x_n, y_n),$$
$$k_2 = h_n \phi(x_n + \frac{h_n}{2}, y_n + \frac{k_1}{2}),$$
$$k_3 = h_n \phi(x_n + \frac{h_n}{2}, y_n + \frac{k_2}{2}),$$
$$k_4 = h_n \phi(x_n + h_n, y_n + k_3).$$



Погрешность приведенной схемы Рунге-Кутта определяется максимальными значением соответствующих производных. Оценку погрешностей можно получить для частного случая вида правой части дифференциального уравнения:
$$\phi(x,\mu) \equiv \phi(x).$$





\section*{Листинги кода}

\lstset{ %
	language=python,                
	basicstyle=\small\sffamily, % размер и начертание шрифта для подсветки кода
	numbers=left,               % где поставить нумерацию строк (слева\справа)
	numberstyle=\tiny,           % размер шрифта для номеров строк
	stepnumber=1,                   % размер шага между двумя номерами строк
	numbersep=5pt,                % как далеко отстоят номера строк от подсвечиваемого кода
	showspaces=false,            % показывать или нет пробелы специальными отступами
	showstringspaces=false,      % показывать или нет пробелы в строках
	showtabs=false,             % показывать или нет табуляцию в строках
	frame=single,              % рисовать рамку вокруг кода
	tabsize=4,                 % размер табуляции по умолчанию равен 2 пробелам
	captionpos=b,              % позиция заголовка вверху [t] или внизу [b] 
	breaklines=true,           % автоматически переносить строки (да\нет)
	breakatwhitespace=false, % переносить строки только если есть пробел
	escapeinside={\#*}{*)}   % если нужно добавить комментарии в коде
}

На листинге 1 приведена реализация метода Рунге-Кутта четвертого порядка на языке Python 3.

\begin{lstlisting}[caption=Методы Рунге-Кутта четвертого порядка]
def runge_kutta_iv(I, U, R, h, Le, Lk, Rk, Ck):
	k1 = f(I, U, R, Le, Lk, Rk)
	m1 = g(I, Ck)
	
	k2 = f(I + h * k1/2, U + h * m1/2, R, Le, Lk, Rk)
	m2 = g(I + h * k1/2, Ck)
	
	k3 = f(I + h * k2/2, U + h * m2/2, R, Le, Lk, Rk)
	m3 = g(I + h * k2/2, Ck)

	k4 = f(I + h * k3, U + h * m3, R, Le, Lk, Rk)
	m4 = g(I + h * k3, Ck)
	
	In = I + h * (k1 + 2*k2 + 2*k3 + k4) / 6 
	Un = U + h * (m1 + 2*m2 + 2*m3 + m4) / 6
	
	return (In, Un)
\end{lstlisting}


На листинге 2 приведена реализация расчета $R_p(T)$ на языке Python 3.

\begin{lstlisting}[caption=Методы Рунге-Кутта четвертого порядка]
def interpolate(table, table_y, y):
	idx_max = 0
	idx_min = 0
	for i in range(len(table_y)):
		if y > table_y[i]:
			idx_max = i
		else:
			idx_max = i
			break
	
	if not idx_max:
		idx_max = 1
	idx_min = idx_max - 1
	
	value = table[idx_min] + (table[idx_max] - table[idx_min]) / (table_y[idx_max] - table_y[idx_min]) * (y - table_y[idx_min)
	return value


def f_I(I, z):
	t0 = interpolate(table_T0, table_I, I)
	m = interpolate(Table_M, table_I, I)
	
	t = t0 + (tw - t0) * (z ** m)
	sigma = interpolate(table_Sigma, table_T, t)
	
	return sigma * z
	

def integ(I):
	a = 0
	b = 1
	n = 100
	h = (b - a) / n
	s = ()f_I(I, a) + f_I(I, b)) / 2
	x = 0
	for i in range(n - 1):
		x += h
		s += f_I(I, x)
	s *= h
	
	return s
	
	
def Rp(I, R, Le):
	return Le / (2*pi * R*R * integ(I)) 
\end{lstlisting}




\section*{Выполнение заданий лабораторной работы}

\subsection*{Графики зависимости от времени импульса t: I(t), U(t), Rp(t), произведения I(t) * Rp(t), T0(t) при заданных выше параметрах. Указать шаг сетки}


На рисунках 1-5 изображены графики зависимости I(t), U(t), Rp(t), произведения I(t) * Rp(t), T0(t).

Шаг сетки: $1/2 * e^{-6}$.

\img{80mm}{1}{График зависимости I(t)}
\img{80mm}{2}{График зависимости U(t)}
\img{80mm}{3}{График зависимости Rp(t)}
\img{80mm}{4}{График зависимости произведения I(t) * Rp(t)}
\img{80mm}{5}{График зависимости T0(t)}


\subsection*{График зависимости I(t) при Rk + Rp = 0}

На рисунке 6 приведен график зависимости $I(t) при R_k + R_p = 0$.

\img{60mm}{6}{График зависимости I(t) при Rk + Rp = 0}


\subsection*{График зависимости I (t) при Rk + Rp = const = 200 Ом в интервале значений t 0-20 мкс}

На рисунке 7 приведен график зависимости $I(t) при R_k + R_p = const = 200$ в интервале значений t 0-20 мкс.

\img{60mm}{7}{График зависимости I(t) при Rk + Rp = const = 200}


\subsection*{Результаты исследования влияния параметров контура Ck, Lk, Rk на длительность импульса t имп. апериодической формы}

Длительность импульса определяется по кривой зависимости тока от времени на высоте $0.35I_{max}$.

Изначальные замеры:
\begin{itemize}
	\item $C_k = 268 * 10^{-6}$ Ф;
	\item $L_k = 187 * 10^{-6} $ Гн;
	\item $R_k = 0.25$ Ом.
\end{itemize}


При заданных параметрах получим следующие величины:
\begin{itemize}
	\item $I_{max} = 788.175$ А, $0.35I_{max} = 275.861$ А;
	\item $t_{imp\_start} = 47,5 * 10^{-6}$ c, $t_{imp\_stop} = 613 * 10^{-6}$ c;
	\item $t_{imp} = t_{imp\_stop} - t_{imp\_start} = 565,5 * 10^{-6}$ c.
\end{itemize}


\subsubsection{Исследование влияния параметра Ck}

Уменьшим значение $C_k$: пусть $C_k = 150 * 10^{-6}$ Ф. При таких параметрах получим:
\begin{itemize}
	\item $I_{max} = 654.55$ А, $0.35I_{max} = 229.092$ А;
	\item $t_{imp\_start} = 38,5 * 10^{-6}$ c, $t_{imp\_stop} = 456 * 10^{-6}$ c;
	\item $t_{imp} = t_{imp\_stop} - t_{imp\_start} = 417,5 * 10^{-6}$ c.
\end{itemize}

Увеличим значение $C_k$: пусть $C_k = 300 * 10^{-6}$ Ф. При таких параметрах получим:
\begin{itemize}
	\item $I_{max} = 838.551$ А, $0.35I_{max} = 293.493$ А;
	\item $t_{imp\_start} = 5,1 * 10^{-6}$ c, $t_{imp\_stop} = 683 * 10^{-6}$ c;
	\item $t_{imp} = t_{imp\_stop} - t_{imp\_start} = 632 * 10^{-6}$ c.
\end{itemize}

На основании проделанных вычислений можно сделать вывод, что длительность импульса прямо пропорциональна $C_k$: при уменьшении $C_k$ длительность импульса уменьшается и, наоборот, при увеличении $C_k$ длительность импульса увеличивается.

\subsubsection{Исследование влияния параметра Lk}

Уменьшим значение $L_k$: пусть $L_k = 100 * 10^{-6}$ Гн. При таких параметрах получим:
\begin{itemize}
	\item $I_{max} = 940.811$ А, $0.35I_{max} = 329,284$ А;
	\item $t_{imp\_start} = 3,1 * 10^{-6}$ c, $t_{imp\_stop} = 455,5 * 10^{-6}$ c;
	\item $t_{imp} = t_{imp\_stop} - t_{imp\_start} = 424,5 * 10^{-6}$ c.
\end{itemize}


Увеличим значение $L_k$: пусть $C_k = 230 * 10^{-6}$ Гн. При таких параметрах получим:
\begin{itemize}
	\item $I_{max} = 738.55$ А, $0.35I_{max} = 258,493$ А;
	\item $t_{imp\_start} = 5,4 * 10^{-6}$ c, $t_{imp\_stop} = 678,5 * 10^{-6}$ c;
	\item $t_{imp} = t_{imp\_stop} - t_{imp\_start} = 624,5 * 10^{-6}$ c.
\end{itemize}

На основании проделанных вычислений можно сделать вывод, что длительность импульса прямо пропорциональна $L_k$ (так же, как и $C_k$): при уменьшении $L_k$ длительность импульса уменьшается и, наоборот, при увеличении $L_k$ длительность импульса увеличивается.


\subsubsection{Исследование влияния параметра Rk}

Уменьшим значение $R_k$: пусть $R_k = 0,1$ Ом. При таких параметрах получим:
\begin{itemize}
	\item $I_{max} = 870.66$ А, $0.35I_{max} = 304,731$ А;
	\item $t_{imp\_start} = 5,2 * 10^{-6}$ c, $t_{imp\_stop} = 607,5 * 10^{-6}$ c;
	\item $t_{imp} = t_{imp\_stop} - t_{imp\_start} = 555,5 * 10^{-6}$ c.
\end{itemize}


Увеличим значение $R_k$: пусть $R_k = 2,2$ Ом. При таких параметрах получим:
\begin{itemize}
	\item $I_{max} = 596.373$ А, $0.35I_{max} = 208,731$ А;
	\item $t_{imp\_start} = 3,6 * 10^{-6}$ c, $t_{imp\_stop} = 667 * 10^{-6}$ c;
	\item $t_{imp} = t_{imp\_stop} - t_{imp\_start} = 631 * 10^{-6}$ c.
\end{itemize}

На основании проделанных вычислений можно сделать вывод, что длительность импульса прямо пропорциональна $R_k$ (так же, как и $C_k$ и $L_k$): при уменьшении $R_k$ длительность импульса уменьшается и, наоборот, при увеличении $R_k$ длительность импульса увеличивается.








\section*{Ответы на контрольные вопросы}


\subsubsection{1. Какие способы тестирования программы, кроме указанного в п.2, можете предложить ещё?}

\begin{enumerate}
	\item Провести тестирование при разном значении шага: если при очередном уменьшении шага результат не изменился, значит найден хороший шаг;
	\item сравнить результаты двух методов разной точности (например, Рунге-Кутта 2-ого и 4-ого порядка точности);
	\item запустить программу на относительно больших значениях сопротивления.
\end{enumerate}


\subsubsection{2. Получите систему разностных уравнений для решения сформулированной задачи неявным методом трапеций. Опишите алгоритм реализации полученных уравнений.}
$$
\left\{\begin{matrix}
	f(I, U) = \frac{dI}{dT} = \frac{U - (R_k + R_p(I))I}{L_k} \\ 
	g(I) = \frac{dU}{dt} = -\frac{I}{C_k}\\
\end{matrix}\right.
$$

Выражения для метода трапеций:
$$
\left\{\begin{matrix}
	I_{n+1} = I_n + \Delta t\frac{f(I_n, U_n) + f(I_{n+1}, U_{n+1})}{2}\\ 
	U_{n+1} = U_n + \Delta t\frac{g(I_n) + g(I_{n+1})}{2}\\
\end{matrix}\right.
$$

$$
\left\{\begin{matrix}
	I_{n+1} = I_n + \Delta t\frac{U_n - (R_k + R_p(I_n))I_n + U_{n+1} - (R_k + R_p(I_{n+1}))I_{n+1}}{2L_k}\\ 
	U_{n+1} = U_n - \Delta t\frac{I_n + I_{n+1}}{2C_k}\\
\end{matrix}\right.
$$

Получили систему уравнений с двумя неизвестными: $I_{n+1}$ и $U_{n+1}$. Подставим $U_{n+1}$ из второго уравнения в первое и решим полученное уравнение относительно $I_{n+1}$:

$$
I_{n+1} = \frac{-2C_k R_p(I_n)\Delta t + 4C_k L_k I_n - 2C_k I_n R_k \Delta t + 4C_k U_n \Delta t - I_n \Delta t^2 }{4C_k L_k + 2C_k R_k \Delta t + 2C_k R_p(I_{n+1}) \Delta t + \Delta t^2}.
$$

Это уравнение решается методом простой итерации, то есть сначала в правую часть подставляется уже известное значение $R_p(I_{n+1})$, затем значение высчитывается повторно, но уже с новым найденным значением.


\subsubsection{3. Из каких соображений проводится выбор численного метода того или иного порядка точности, учитывая, что чем выше порядок точности метода, тем он более сложен и требует, как правило, больших ресурсов вычислительной системы?}

В случае, если функция $\phi(x, \mu)$  ограничена и непрерывна, а также ограничены и непрерывны ее N-ые производные, то оптимальный результат достигается при применении метода N-ого порядка точности. В случае,если же функция $\phi(x, \mu)$ имеет меньше N производных, то и N-ый порядок точности не сможет быть достигнут. Очевидно, тратить вычислительные ресурсы системы на расчет метода N-ого порядка точности не имеет смысла, логичнее будет использование некоторый метод меньшей точности.


\subsubsection{4. Можно ли метод Рунге-Кутта применить для решения задачи, в которой часть условий задана на одной границе, а часть на другой? Например, напряжение по-прежнему задано при t=0, т.е. t=0, U=U0, а ток задан в другой момент времени, к примеру, в конце импульса, т.е. при t=T, I=IT. Какой можете предложить алгоритм вычислений?}

Поскольку в методе Рунге-Кутта для вычисления каждого из последующих значений необходимо знать значения обеих параметров (в нашем примере, силы тока и напряжения) в предыдущий момент времени, то сделаю вывод, что нет, в таком случае нельзя использовать метод Рунге-Кутта.

Для решения подобной задачи необходимо использовать алгоритм, способный решать краевые задачи и уже внутри него можно использовать, например, метод Рунге-Кутта заданного порядка точности.

