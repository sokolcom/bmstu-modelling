\chapter{Аналитическая часть}

\section{Сортировка вставками}

На каждой итерации алгоритма выбирается один из элементов неотсортированной части массива (максимальный либо минимальный  в зависимости от направления сортировки) 
и помещается на нужную позицию в отсортированную часть массива. \cite{Cormen} Таким образом происходит вставка элемента на нужную позицию.



\section{Сортировка пузырьком}

Алгоритм проходит по массиву n-1 раз или до тех пор, пока массив не будет полностью отсортирован. 
В каждом проходе элементы попарно сравниваются и, при необходимости, меняются местами.
При каждом проходе алгоритма по внутреннему циклу, очередной наибольший элемент ставится на своё место в конец неотсортированного массива. \cite{Cormen}
Таким образом наибольшие элементы "всплывают" как пузырек, отчего. собственно, и произошло название данного алгоритма. 



\section{Быстрая сортировка}

Алгоритм быстрой сортировки (сортировки Хоара) заключается в следующем:
\begin{enumerate}
	\item массив разбивается на два (возможно пустых) подмассива таких, что в одном подмассиве каждый элемент меньше либо равен опорному, 
	и при этом не превышает любой элемент второго подмассива;
	\item подмассивы сортируются с помощью рекурсивного вызова процедуры быстрой сортировки.
\end{enumerate} 

Стоит отметить, что поскольку подмассивы сортируются на месте, для их объединения не трубуются никакие действия. \cite{Cormen}



\section*{Вывод}
В этом разделе были рассмотрены алгоритмы сортировки вставками, пузырьком и быстрой сортировки. Вычислительная сложность данных алгоритмов будет рассмотрена в следующем разделе.
