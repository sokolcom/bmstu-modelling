
\section*{Цель работы}
Целью данной лабораторной работы является анализ и сравнение численных методов (методов Эйлера и Рунге-Кутта 2-ого порядка) и приближенного аналитического метода Пикара для решения задачи Коши ОДУ. 




\section*{Задача Коши}
Задача Коши определяется следующей системой:
$$
\left\{\begin{matrix}
	u'(x) = f(x, u)\\ 
	u(\varepsilon) = \eta \\
\end{matrix}\right.
$$

Решение данной задачи можно получить с помощью приближенных аналитических или численных методов.


\section*{Метод Пикара}

i-ое приближение рассчитывается по следующей формуле:\\
$$y^{(i)}(x) = \eta + \int_{0}^{x}{f(t, y^{(i-1)}(t))dt}$$

В данной лабораторной работе 
$$
\left\{\begin{matrix}
	u'(x) = u^2 + x^2\\ 
	u(0) = 0 \\
\end{matrix}\right.
$$

Вычислим первые 4 приближения метода Пикара:
$$y^{(1)}(x) = 0 + \int_{0}^{x}{t^2 dt} = \frac{x^3}{3}$$\\
$$y^{(2)}(x) = 0 + \int_{0}^{x}{y^{(1)}dt} = y^{(1)} [ \frac{x^4}{21} + 1]$$\\
$$y^{(3)}(x) = 0 + \int_{0}^{x}{y^{(2)}dt} = y^{(1)} [ \frac{x^{12}}{19845} + \frac{2x^8}{693}] + y^{(2)}$$\\
$$y^{(4)}(x) = 0 + \int_{0}^{x}{y^{(3)}dt} = y^{(1)} [ \frac{x^{28}}{36 625 634 325} + \frac{4x^{2}}{1 113 959 385} + \frac{662x^{20}}{3 479 404 005} + $$
$$ + \frac{82x^{16}}{12 442 815} + \frac{4x^{12}}{31185}] + y^{(3)}$$


\section*{Метод Эйлера}

Порядок точности метода: $O(h)$.
$$y_{n+1} = y_n + h * f(x_n, y_n)$$


\section*{Метод Рунге-Кутта 2-ого порядка}

Порядок точности метода: $O(h^2)$.
$$y_{n+1} = y_n + h * [(1 - \alpha)k_1 + \alpha k_2],$$
где $\alpha = \overline{0, 0.5}$, $k_1 = f(x_n, y_n)$, $k_2 = f(x_n + \frac{h}{2\alpha}, y_n + \frac{h}{2\alpha}k_1)$




\section*{Листинги кода}

\lstset{ %
	language=python,                
	basicstyle=\scriptsize\sffamily, % размер и начертание шрифта для подсветки кода
	numbers=left,               % где поставить нумерацию строк (слева\справа)
	numberstyle=\tiny,           % размер шрифта для номеров строк
	stepnumber=1,                   % размер шага между двумя номерами строк
	numbersep=5pt,                % как далеко отстоят номера строк от подсвечиваемого кода
	showspaces=false,            % показывать или нет пробелы специальными отступами
	showstringspaces=false,      % показывать или нет пробелы в строках
	showtabs=false,             % показывать или нет табуляцию в строках
	frame=single,              % рисовать рамку вокруг кода
	tabsize=4,                 % размер табуляции по умолчанию равен 2 пробелам
	captionpos=b,              % позиция заголовка вверху [t] или внизу [b] 
	breaklines=true,           % автоматически переносить строки (да\нет)
	breakatwhitespace=false, % переносить строки только если есть пробел
	escapeinside={\#*}{*)}   % если нужно добавить комментарии в коде
}

На листингах 1-3 реализованных методов на языке Python 3.

\begin{lstlisting}[caption=Методы Пикара]
def picard_1(x):
	"""Picard's method (1st approximation)"""
	return x ** 3 / 3


def picard_2(x):
	"""Picard's method (2nd approximation)"""
	return picard_1(x) * ((x ** 4) * (1 / 21) + 1)


def picard_3(x):
	"""Picard's method (3rd approximation)"""
	return picard_1(x) * ((x ** 12) * (1 / 19845) + 2 * (x ** 8) * (1 / 693)) + picard_2(x)


def picard_4(x):
	"""Picard's method (4th approximation)"""
	return picard_1(x) * ((x ** 28) * (1 / 36_625_634_325) + 4 * (x ** 24) * (1 / 1_113_959_385) + 662 * (x ** 20) * (1 / 3_479_404_005) + \
	82 * (x ** 16) * (1 / 12_442_815) + 4 * (x ** 12) * (1 / 31185)) + picard_3(x)
\end{lstlisting}

\begin{lstlisting}[caption=Метод Эйлера]
def euler_method(x_n, y_n, h):
	"""Euler's method (explicit scheme)"""
	return y_n + h * f(x_n, y_n)
\end{lstlisting}

\begin{lstlisting}[caption=Методы Рунге-Кутта второго порядка]
def runge_kutta_method(x_n, y_n, h):
	"""Runge-Kutta method (2nd order)"""

	alpha = 1  # alpha=1 or alpha=1/2
	k1 = f(x_n, y_n)
	k2 = f(x_n + h / (2 * alpha), y_n + h / (2 * alpha) * k1)
	return y_n + h * ((1 - alpha) * k1 + alpha * k2)
\end{lstlisting}


