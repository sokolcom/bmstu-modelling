\chapter{Технологическая часть}
\section{Выбор ЯП}
Для реализации программы был выбран язык программирования Python 3 в следствие относительной простоты написание кода на данном ЯП. \cite{python} Среда разработки - редактор кода VS Code. \cite{vscode}

\section{Листинги кода алгоритмов}


Далее будут приведены листинги алгоритмов сортировки вставками (листинг 3.1), алгоритма сортировки пузырьком (листинг 3.2) и алгоритм быстрой сортировки (листинг 3.3).

% Для листинга кода:
\lstset{ %
	language=python,                 % выбор языка для подсветки 
	basicstyle=\scriptsize\sffamily, % размер и начертание шрифта для подсветки кода
	numbers=left,               % где поставить нумерацию строк (слева\справа)
	numberstyle=\tiny,           % размер шрифта для номеров строк
	stepnumber=1,                   % размер шага между двумя номерами строк
	numbersep=5pt,                % как далеко отстоят номера строк от подсвечиваемого кода
	showspaces=false,            % показывать или нет пробелы специальными отступами
	showstringspaces=false,      % показывать или нет пробелы в строках
	showtabs=false,             % показывать или нет табуляцию в строках
	frame=single,              % рисовать рамку вокруг кода
	tabsize=2,                 % размер табуляции по умолчанию равен 2 пробелам
	captionpos=t,              % позиция заголовка вверху [t] или внизу [b] 
	breaklines=true,           % автоматически переносить строки (да\нет)
	breakatwhitespace=false, % переносить строки только если есть пробел
	escapeinside={\#*}{*)}   % если нужно добавить комментарии в коде
}


\begin{lstlisting}[label=CodeStand,caption=Алгоритм сортровки вставками]
def insertion_sort(array):
	for j in range(1, len(array)):
		i = j - 1
		value = array[j]
	
		while (i >= 0) and (array[i] > value):
			array[i + 1] = array[i]
			i -= 1
		
	array[i + 1] = value
\end{lstlisting}


\begin{lstlisting}[label=some-code,caption=Алгоритм сортировки пузырьком]
def bubble_sort(array):
	length = len(array)
	
	for i in range(length - 1):
		for j in range(i + 1, length):
			if array[i] > array[j]:
				temp = array[i]
				array[i] = array[j]
				array[j] = temp 
\end{lstlisting}


\begin{lstlisting}[label=some-code,caption=Алгоритм быстрой сортировки]
def qsort(array, first, last):
	if first < last:
		f = first
		l = last
		base = array[(first + last) // 2]
		
		while f <= l:
			elem_f = array[f]
			while (elem_f < base) and (f <= last):
				f += 1
				elem_f = array[f]
			
			elem_l = array[l]
			while (elem_l > base) and (l >= first):
				l -= 1
				elem_l = array[l]
			
			if f <= l:
				temp = array[f]
				array[f] = array[l]
				array[l] = temp
				f += 1
				l -= 1
		qsort(array, first, l)
		qsort(array, f, last)
\end{lstlisting}


\section{Тесты}


\par
Проведем тестирование программы в таблице 3.1.

В столбце "Исходный массив" и "Отсорт. массив" приведены исходные массивы для сортировки и ожидаемые отсортированные массивы. В столбце Алг. №1 содержится результат выполнения алгоритма сортировки вставками, в Алг. №2 - алгоритма сортировки пузырьком, а в Алг. №3 - быстрой сортировки соответственно. Сортировать будем по возрастанию.


\begin{table}[h!]
	\centering
	\caption{Таблица тестовых данных}
	\begin{tabular}{|c c c c c c|} 
		\hline
		№ & Исходный массив & Отсорт. массив & №1 & №2 & №3 \\ [0.8ex] 
		\hline
		1 & $[ ]$ & $[ ]$ & $[ ]$ &  $[ ]$ & $[ ]$ \\
		\hline
		2 & $[1, 2, 3]$ & $[1, 2, 3]$ & $[1, 2, 3]$ &  $[1, 2, 3]$ & $[1, 2, 3]$ \\
		\hline
		3 & $[3, 2, 1]$ & $[1, 2, 3]$ & $[1, 2, 3]$ &  $[1, 2, 3]$ & $[1, 2, 3]$ \\
		\hline
		4 & $[2, 3, 1]$ & $[1, 2, 3]$ & $[1, 2, 3]$ &  $[1, 2, 3]$ & $[1, 2, 3]$ \\
		\hline
		5 & $[2, 1, 3, 1]$ & $[1, 1, 2, 3]$ & $[1, 1, 2, 3]$ &  $[1, 1, 2, 3]$ & $[1, 1, 2, 3]$ \\
		\hline
	\end{tabular}
\end{table}


\section*{Вывод}
В данном разделе была рассмотрена структура ПО и листинги кода программы; также убедились в корректности работы разработанного ПО.




%%%%%%%%%%%%%%%%%%%%%%%%%%%%%%%%%%%%%%%%%%%%%%%%%%%%%%%%%%%%%%%%%%%%%%%



\chapter{Исследовательская часть}

\section{Сравнение алгоритмов}

Был проведен замер времени работы каждого из алгоритмов на машине HP ProBook 450 G5. \cite{computer} 

В рамках данного эксперимента было произведено сравнение времени выполнения трех алгоритмов в лучшем/худшем/случайном случае заполнения массива. При длине массивов от 100 до 1000 элементов с шагом 100. На графике 4.1 показано сравнение времени в лучшем случае,  на графике 4.2 - сравнение времени в худшем случае, на графике 4.3 сравнение времени при случайном заполнении массива. По оси l идет длина массива, а по оси t - время сортировки в секундах.  Для минимизации погрешности замеров времени каждый алгоритм исполнялся над одними и теми же строками 100 раз и затраченное время делилось на 100, для получения усредненного времени выполнения.



\begin{tikzpicture}
	\begin{axis}[
		axis lines = left,
		xlabel = $l$,
		ylabel = {$t$, сек.},
		legend pos=north west,
		ymajorgrids=true
		]
		\addplot[color=red, mark=square] table[x index=0, y index= 1] {insertion_asc.txt}; 
		\addplot[color=blue, mark=square] table[x index=0, y index= 1] {bubble_asc.txt}; 
		\addplot[color=green, mark=square] table[x index=0, y index= 1] {qsort_asc.txt}; 
		
		\addlegendentry{С. вставками}
\addlegendentry{С. пузырьком}
\addlegendentry{Быстрая с.}
	\end{axis}
\end{tikzpicture}
\begin{center}
	Pис. 4.1: Сравнение времени выполнения в лучшем случае
\end{center}


\begin{tikzpicture}
	\begin{axis}[
		axis lines = left,
		xlabel = $l$,
		ylabel = {$t$, сек.},
		legend pos=north west,
		ymajorgrids=true
		]
		\addplot[color=red, mark=square] table[x index=0, y index= 1] {insertion_desc.txt}; 
		\addplot[color=blue, mark=square] table[x index=0, y index= 1] {bubble_desc.txt}; 
		\addplot[color=green, mark=square] table[x index=0, y index= 1] {qsort_desc.txt}; 
		
		\addlegendentry{С. вставками}
\addlegendentry{С. пузырьком}
\addlegendentry{Быстрая с.}
	\end{axis}
\end{tikzpicture}
\begin{center}
	Pис. 4.2: Сравнение времени выполнения в худшем случае
\end{center}



\begin{tikzpicture}
	\begin{axis}[
		axis lines = left,
		xlabel = $l$,
		ylabel = {$t$, сек.},
		%ymax=0.0005,
		legend pos=north west,
		ymajorgrids=true
		]
		\addplot[color=red, mark=square] table[x index=0, y index= 1] {insertion_rnd.txt}; 
		\addplot[color=blue, mark=square] table[x index=0, y index= 1] {bubble_rnd.txt}; 
		\addplot[color=green, mark=square] table[x index=0, y index= 1] {qsort_rnd.txt}; 
		
		\addlegendentry{С. вставками}
		\addlegendentry{С. пузырьком}
		\addlegendentry{Быстрая с.}
	\end{axis}
\end{tikzpicture}
\begin{center}
	Pис. 4.3: Сравнение времени выполнения при случайном заполнении массива
\end{center}



\section*{Вывод}

Рассматривая графики затрат времени при худшем случае алгоритм быстрой сортировки показывает наихудшие результаты: он медленнее примерно в 10 раз сортировки пузырьком. При худшем случае ситуация обратная: тут зависимость видна исходя их графиков функции $n*log(n)$ для быстрой сортировки и $n^2$ для сортировок пузырьком и выбором. Для случайного заполнения массива ситуация аналогична лучшему случаю.

Сравнивая алгоритмы пузырьковой сортировки и сортировки выбором можно заметить, что сортировка вставками работает быстрее за счет меньшего коэффициента при старшей степени ($n^2$).










	
	









