
\section*{Задание}
В многофункциональный центр приходят посетители через интервал времени 5 +- 2 минуты. При входе производится проверка наличия маски в течение 1 минуты. Если у посетителя нет маски, ему отказывается в обслуживании. Вероятность отсутствия маски у посетителя составляет 5\%. После проверки наличия маски производится проверка температуры в течение 3 +- 1 минуты. С вероятностью 2\% у посетителя будет температура и ему будет отказано в обслуживании. Далее посетители проходят к терминалам для получения талона очереди в течение 4 +- 1 минуты. Вероятность того, что в многофункциональный центр не предоставляют необходимую услугу, равняется 5\% Если все три терминала заняты, ему будет отказано в обслуживании. Если в очереди на окно набралось 10 человек, посетителю 
отказывают. Всего есть 3 окна, которые работают 15 +- 5 , 10 
+- 2 , 20 +- 5 минут соответственно. Окно на терминале выбирается по равномерному распределению.

Промоделировать процесс обработки 800 запросов. Определить вероятность отказа.



\section*{Схема системы}
На рисунке 1 приведена схема моделируемой системы.
\img{45mm}{1}{Схема системы}


\section*{Листинги}
% Для листинга кода:
\lstset{ %
	language=python,                 % выбор языка для подсветки 
	basicstyle=\scriptsize\sffamily, % размер и начертание шрифта для подсветки кода
	numbers=left,               % где поставить нумерацию строк (слева\справа)
	numberstyle=\tiny,           % размер шрифта для номеров строк
	stepnumber=1,                   % размер шага между двумя номерами строк
	numbersep=5pt,                % как далеко отстоят номера строк от подсвечиваемого кода
	showspaces=false,            % показывать или нет пробелы специальными отступами
	showstringspaces=false,      % показывать или нет пробелы в строках
	showtabs=false,             % показывать или нет табуляцию в строках
	frame=single,              % рисовать рамку вокруг кода
	tabsize=2,                 % размер табуляции по умолчанию равен 2 пробелам
	captionpos=t,              % позиция заголовка вверху [t] или внизу [b] 
	breaklines=true,           % автоматически переносить строки (да\нет)
	breakatwhitespace=false, % переносить строки только если есть пробел
	escapeinside={\#*}{*)}   % если нужно добавить комментарии в коде
}



На листинге 1 приведена реализация класса Model; с помощью метода event\_mode выполняется моделирование системы.
\begin{lstlisting}[label=CodeStand,caption=Класс Model]
class Model:
	def __init__(self, clients, mask_check, temp_check, terminals, windows):
		self.clients = clients
		self.mask_check = mask_check
		self.temp_check = temp_check
		self.terminals = terminals
		self.windows = windows

	def event_mode(self):
		clients = self.clients
		mask = self.mask_check
		
		clients.receivers = self.mask_check.copy()
		self.mask_check[-1].receivers = self.temp_check.copy()
		self.temp_check[-1].receivers = self.terminals.copy()
		
		self.terminals[0].receivers = self.windows.copy()
		self.terminals[1].receivers = self.windows.copy()
		self.terminals[2].receivers = self.windows.copy()
		
		clients.next = clients.delay()
	
		blocks = []
		blocks += [clients]
		blocks += self.mask_check
		blocks += self.temp_check
		blocks += self.terminals
		blocks += self.windows
		
		refusals = {}
		
		for block in blocks:
			refusals[block.name] = 0
		refusals["query_window"] = 0
	
		while clients.num_requests >= 0:
			current_time = clients.next
			for block in blocks:
				if (0 < block.next) and (block.next < current_time):
					current_time = block.next
	
		for block in blocks:
			if current_time == block.next:
				if not isinstance(block, ProcessRequest):
					next_clients = clients.generate_request()
					if next_clients is not None:
						next_clients.next = current_time + next_clients.delay()
					else:
						refusals[block.name] += 1
					clients.next = current_time + clients.delay()
				else:
					next_process = block.process_request()
					if block.queue == 0:
						block.next = 0
					else:
						block.next = current_time + block.delay()
		
					if next_process == "kick":
						if block.name.find("terminal") != -1:
							refusals["query_window"] += 1
						else:
							refusals[block.name] += 1
						continue
		
					if block.end:
						continue
		
					if next_process is not None:
						next_process.next = current_time + next_process.delay()
					else:
						refusals[block.name] += 1
		
		return refusals
\end{lstlisting}



\section*{Результаты выполнения работы}

На рисунке 2 приведен результат работы системы с 800 заявок.
\img{170mm}{2}{Результат моделирования системы с 800 заявками}







