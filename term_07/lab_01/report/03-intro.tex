
\section*{Цель работы}
Целью данной лабораторной работы является исследования функций распределения и функций плотности распределения случайных величин. Необходимо исследовать 2 распределения: равномерное и второе - по варианту.

\textbf{Вариант №15}: 15 mod 4 = 3 => необходимо исследовать нормальное распределение.




\section*{Равномерное распределение}
Равномерное распределение непрерывной случайной величины – это распределение, в котором значения случайной величины с двух сторон ограничены и в границах интервала имеют одинаковую вероятность.
Плотность вероятности в данном интервале постоянна. Равномерное
распределение обозначают $X \sim R(a, b)$, где $a, b \in \mathbb{R}$.

Функция плотности $f_{X} (x)$ имеет вид:
$$
f_{X} (x) = 
\left\{\begin{matrix}
	\frac{1}{b - a}, x \in [a, b]\\ 
	0, x \notin [a, b] \\
\end{matrix}\right.
$$

Проинтегрировав функцию плотности, получим функция распределения:
$$
F_{X} (x) = 
\left\{\begin{matrix}
	0, x < a \\
	\frac{x - a}{b - a}, a \le x < b\\ 
	1, x \ge b \\
\end{matrix}\right.
$$

На рисунке 1 приведены графики функций плотности и распределения равномерно распределенной непрерывной случайной величины.

\img{60mm}{1}{Графики плотности и распределения равномерной непрерывной случайной величины}



\section*{Нормальное распределение}

Случайная величина имеет нормальное распределение (обозначается $X \sim N(\mu, \sigma^2)$), если функция ее плотности имеет следующий вид:
$$
f_{X} (x) = \frac{1}{\sigma\sqrt{2\pi}}e^{\frac{(x - \mu)^2}{2\sigma^2}},
$$
где
$$x, \mu \in \mathbb{R}, \sigma > 0$$

Проинтегрировав функцию плотности, получим функция распределения:

$$
F_{X} (x) = \frac{1}{\sigma\sqrt{2\pi}} \int_{-\infty}^{x}{e^{\frac{(x - \mu)^2}{2\sigma^2}}dx},
$$

На рисунке 2 приведены графики функций плотности и распределения нормально распределенной непрерывной случайной величины.

\img{30mm}{2}{Графики плотности и распределения равномерной непрерывной случайной величины}



Математическое ожидание ($\mu$) характеризует положение «центра
тяжести» вероятностной массы нормального распределения. График плотности распределения нормальной случайной величины симметричен относительно прямой $x = \mu$.

Дисперсия ($\sigma^2$) характеризует разброс значений случайной величины
относительно мат. ожидания.



\section*{Результат выполнения задания}

\subsection*{Равномерное распределение}
На рисунке 3 приведены графики функций плотности и распределения равномерно распределенной величины с различными параметрами.

\img{130mm}{3}{Графики плотностей и распределений равномерных непрерывных случайных величин}

\subsection*{Нормальное распределение}
На рисунке 4 приведены графики функций плотности и распределения нормально распределенной величины с различными параметрами.

\img{130mm}{4}{Графики плотностей и распределений нормальных непрерывных случайных величин}




\section*{Вывод}

В результате выполнения лабораторной работы с использованием программных средств были построены графики равномерного и нормального распределений, а также их функции плотности. Также было проведено сравнение этих графиков при разных значениях параметров распределений ($a, b$ для равномерного распределения и $\mu, \sigma$ для нормального).


