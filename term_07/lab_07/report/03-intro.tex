
\section*{Задание}
В информационный центр приходят клиенты через интервал времени 10 +- 2 минуты. Если все три имеющихся оператора заняты, клиенту отказывают в обслуживании. Операторы имеют разную производительность и могут обеспечивать обслуживание среднего запроса пользователя за 20 +- 5; 40 +- 10; 40 +- 20. Клиенты стремятся занять свободного оператора с максимальной производительностью. Полученные запросы сдаются в накопитель. Откуда выбираются на обработку. На первый компьютер запросы от 1 и 2-ого операторов, на второй – запросы от 3-его. Время обработки запросов первым и 2-м компьютером равны соответственно 15 и 30 мин. Промоделировать процесс обработки 300 запросов.

На рисунке 1 приведена схема системы.
\img{35mm}{1}{Схема системы}

Для выполнения поставленного задания необходимо создать концептуальную модель в терминах СМО, определить эндогенные и экзогенные переменные и уравнения модели. За единицу системного времени выбрать 0,01 минуты.



\section*{Теоретическая часть}
В процессе взаимодействия клиентов с информационным центром возможно:
\begin{itemize}
	\item режим нормального обслуживания, т.е. клиент выбирает одного из свободных операторов, отдавая предпочтение тому у которого меньше номер;
	\item режим отказа в обслуживании клиента, когда все операторы заняты. 
\end{itemize}


\subsection*{Переменные и уравнения имитационной модели}
Эндогенные переменные отвечают за время обработки задания i-ым оператором, время решения этого задания j-ым компьютером.

Экзогенные переменные - это число клиентов, которых обслужили и получившие отказ.


\subsection*{Концептуальная схема}

На рисунке 2 приведена концептуальная схема системы в терминах СМО.
\img{70mm}{2}{Концептуальная схема в терминах СМО}

$$
P_{react} = \frac{C_{react}}{C_{react} + C_{proc}}
$$



\section*{Листинги}


На листинге 1 приведен листинг программы на языке GPSS.
\VerbatimInput{code.txt}







\section*{Результаты выполнения работы}

На рисунках 3-5 приведен отчет о результатах работы системы с 300 заявок.
\img{120mm}{3}{Результат моделирования системы с 300 заявками (часть 1)}
\img{120mm}{4}{Результат моделирования системы с 300 заявками (часть 2)}
\img{70mm}{5}{Результат моделирования системы с 300 заявками (часть 3)}


При моделировании системы с 300 заявками процент потерянных заявок составляет 23\%.








