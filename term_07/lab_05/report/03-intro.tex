
\section*{Задание}
В информационный центр приходят клиенты через интервал времени 10 +- 2 минуты. Если все три имеющихся оператора заняты, клиенту отказывают в обслуживании. Операторы имеют разную производительность и могут обеспечивать обслуживание среднего запроса пользователя за 20 +- 5; 40 +- 10; 40 +- 20. Клиенты стремятся занять свободного оператора с максимальной производительностью. Полученные запросы сдаются в накопитель. Откуда выбираются на обработку. На первый компьютер запросы от 1 и 2-ого операторов, на второй – запросы от 3-его. Время обработки запросов первым и 2-м компьютером равны соответственно 15 и 30 мин. Промоделировать процесс обработки 300 запросов.

На рисунке 1 приведена схема системы.
\img{35mm}{1}{Схема системы}

Для выполнения поставленного задания необходимо создать концептуальную модель в терминах СМО, определить эндогенные и экзогенные переменные и уравнения модели. За единицу системного времени выбрать 0,01 минуты.



\section*{Теоретическая часть}
В процессе взаимодействия клиентов с информационным центром возможно:
\begin{itemize}
	\item режим нормального обслуживания, т.е. клиент выбирает одного из свободных операторов, отдавая предпочтение тому у которого меньше номер;
	\item режим отказа в обслуживании клиента, когда все операторы заняты. 
\end{itemize}


\subsection*{Переменные и уравнения имитационной модели}
Эндогенные переменные отвечают за время обработки задания i-ым оператором, время решения этого задания j-ым компьютером.

Экзогенные переменные - это число клиентов, которых обслужили и получившие отказ.


\subsection*{Концептуальная схема}

На рисунке 2 приведена концептуальная схема системы в терминах СМО.
\img{70mm}{2}{Концептуальная схема в терминах СМО}

$$
P_{react} = \frac{C_{react}}{C_{react} + C_{proc}}
$$



\section*{Листинги}
% Для листинга кода:
\lstset{ %
	language=python,                 % выбор языка для подсветки 
	basicstyle=\scriptsize\sffamily, % размер и начертание шрифта для подсветки кода
	numbers=left,               % где поставить нумерацию строк (слева\справа)
	numberstyle=\tiny,           % размер шрифта для номеров строк
	stepnumber=1,                   % размер шага между двумя номерами строк
	numbersep=5pt,                % как далеко отстоят номера строк от подсвечиваемого кода
	showspaces=false,            % показывать или нет пробелы специальными отступами
	showstringspaces=false,      % показывать или нет пробелы в строках
	showtabs=false,             % показывать или нет табуляцию в строках
	frame=single,              % рисовать рамку вокруг кода
	tabsize=2,                 % размер табуляции по умолчанию равен 2 пробелам
	captionpos=t,              % позиция заголовка вверху [t] или внизу [b] 
	breaklines=true,           % автоматически переносить строки (да\нет)
	breakatwhitespace=false, % переносить строки только если есть пробел
	escapeinside={\#*}{*)}   % если нужно добавить комментарии в коде
}



На листинге 1 приведена функция обработки инкремента шага по времени.
\begin{lstlisting}[label=CodeStand,caption=Обработка инкремента шага по времени]
def one_step(generator, operators, processors, request_info, generate_new=True):
	if generate_new:
		request = generator.update_time(time_unit)
		if request:
			request_info['gen'] += 1
			i_operator = pick_operator(operators)
			if i_operator == -1:
				request_info['loss'] += 1
			else:
				operators[i_operator].accept_request(request)
	
	for cur_operator in operators:
		cur_operator.update_time(time_unit)
	
	for cur_processor in processors:
		res = cur_processor.update_time(time_unit)
		if res == 'request done':
			request_info['ok'] += 1
\end{lstlisting}


На листинге 2 приведена код цикла, обеспечивающие пошаговую работу (моделирования) системы.
\begin{lstlisting}[label=CodeStand,caption=Цикл пошагового моделирования системы]
def modeling(generator, operators, processors, total_incoming_requests):
	request_info = { 'gen': 0, 'loss': 0, 'ok': 0 }
	
	while request_info['gen'] < total_incoming_requests:
		one_step(generator, operators, processors, request_info)
	
	while request_info['loss'] + request_info['ok'] < total_incoming_requests:
		one_step(generator, operators, processors, request_info, False)
	
	return request_info
\end{lstlisting}


На листинге 3 приведена код функции main: задание входных параметров и получение результата моделирования.
\begin{lstlisting}[label=CodeStand,caption=Фукнция main]
def main():
	client_generator = Generator(EvenDistribution(8, 12))
	
	first_queue = []
	second_queue = []
	
	operators = [
	Operator(first_queue, EvenDistribution(15, 25)),
	Operator(first_queue, EvenDistribution(30, 50)),
	Operator(second_queue, EvenDistribution(20, 60))
	]
	
	processors = [
	Processor(first_queue, EvenDistribution(15, 15)),
	Processor(second_queue, EvenDistribution(30, 30))
	]
	
	total_requests = 300
	
	t_start = time()
	res = modeling(client_generator, operators, processors, total_requests)
	
	print('Time (seconds)', time() - t_start)
	for key in res.keys():
		print(f"\t{key}\t---\t{res[key]}")
	
	print('Loss rate', res['loss'] / total_requests)
\end{lstlisting}



\section*{Результаты выполнения работы}

На рисунках 3-5 приведены результаты работы системы с 300, 1000, 5000 заявок.
\img{50mm}{3}{Результат моделирования системы с 300 заявками}
\img{50mm}{4}{Результат моделирования системы с 1000 заявками}
\img{50mm}{5}{Результат моделирования системы с 5000 заявками}


При моделировании системы с 300 заявками процент потерянных заявок составляет 22\%.








