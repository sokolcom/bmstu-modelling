\chapter*{Постановка задачи}
\addcontentsline{toc}{chapter}{Постановка задачи}

Проектирование, разработка и поддерживание соревновательной системы для оттачивания навыков программирования, созданная для студентов кафедры ИУ7 МГТУ им. Н.Э. Баумана, проходящих курс <<Программирование на СИ>>.

Внедрение данной соревновательной системы для прохождения студентами 1-го курса летней практики по направлению <<Углубленный СИ>>.

Основными требованиями к системе являются: 

\begin{enumerate}[label={\arabic*)}]
	\item обучение языку программирования в процессе выполнения заданий соревновательной системы;
	\item минимальное количество действий со стороны обучающегося при использовании системы;
	\item возможность принять участие в соревнованиях для любого желающего;
	\item определенность правил и распределения очков в рейтинге между студентами;
	\item масштабирование рейтинговой системы на целевую аудиторию.
\end{enumerate}

Ограничения, наложенные на проект: 

\begin{enumerate}[label={\arabic*)}]
	\item реализовать всё ПО используя язык программирования Python \cite{python};
	\item реализовать рейтинговую систему не используя базы данных \cite{db};
	\item соревновательная система поддерживает только язык программирования СИ;
	\item система работает для обучающихся, имеющих аккаунт на кафедральном GitLab \cite{gitlab}.
\end{enumerate}