
\section*{Задание}
Изучить методы генерирования псевдослучайных чисел, а также критерии оценки случайности последовательности. Реализовать критерий оценки случайной последовательности. Сравнить результаты работы данного критерия на одноразрядных, двухразрядных и трехразрядных последовательностях целых чисел. Последовательности получить алгоритмическим и табличным способами.


\section*{Теоретическая часть}
Выбранный алгоритм генерации случайных чисел - линейный конгруэнтный генератор.

Выбранный критерий оценки случайности последовательности: критерий <<Хи-квадрат>>. Это один из самых распространенных статистических критериев, иногда этот метод используется в сочетании с другими критериями. С помощью этого критерия можно оценить, удовлетворяет ли генератор случайных чисел
требованию равномерного распределения или нет.

Для оценки по этому критерию необходимо вычислить статистику $V$ по формуле:
$$
V = \frac{1}{n} \sum_{s=1}^{k} \frac{Y_{s}^{2}}{p_{s}} - n,
$$
где
\begin{itemize}
	\item $n$ - количество независимых испытаний;
	\item $k$ - количество категорий;
	\item $Y_s$ - число наблюдений, которые действительно относятся к категории $s$;
	\item $p_s$ - вероятность того, что каждое наблюдение относится к категории $s$.
\end{itemize}

Значение $V$ является значением критерия <<Хи-квадрат>> для экспериментальных данных. Приемлемое значение этого критерия можно определить по таблице 1. Для этого используем
строку с v = k-1, где k = 10, 90, 900 для задания лабораторной. P в этой таблице — это
вероятность того, что экспериментальное значение $V_{exp}$. будет меньше теоретического
$V_{theor}$. или равно ему. Также ее можно рассматривать как доверительную вероятность.


\begin{table}[h!]
	\centering
	\caption{Значения $V_{theor}$ для количества степеней свободы по заданию}
	\begin{tabular}{| c | p{1.5cm} | p{1.5cm} | p{1.5cm} | p{1.5cm} | p{1.5cm} | p{1.5cm} | p{2cm} |} 
		\hline
		$n - 1$ & P = 1\% &  P = 5\% &  P = 25\% &  P = 50\% &  P = 75\% &  P = 95\% &  P = 99\% \\ [0.5ex] 
		\hline
		V = 9 & 2.088 & 3.325 & 5.899 & 8.343 & 11.39 & 16.92 & 21.67 \\ [0.5ex] 
		\hline
		V = 89 & 60.93 & 68.25 & 79.68 & 88.33 & 97.60 & 112.02 & 122.94 \\ [0.5ex] 
		\hline
		V = 899 & 803.31 & 830.41 & 870.05 & 898.33 & 927.23 & 969.86 & 1000.57 \\ [0.5ex]  
		\hline
	\end{tabular}
\end{table}


Если вычисленное $V$ окажется меньше 1\%-й точки или больше 99\%-й точки, можно сделать
вывод, что эти числа недостаточно случайные. Если $V$ лежит между 1\% и 5\% точками или
между 95\% и 99\% точками, то эти числа <<подозрительны>>. Если $V$ лежит между 5\% и 10\%
точками или 90\%-95\% точками, то числа можно считать <<почти подозрительными>>. Обычно
необходимо произвести проверку три раза и более с разными данными. Если по крайней мере
два из трех результатов оказываются подозрительными, то числа рассматриваются как
недостаточно случайные.


Таким образом, процедура проверки критерия свдится к следующему алгоритму:
\begin{enumerate}
	\item выделяем $k$ категорий; в нашем случае это количество возможных полученных
	значений: 10, 90 и 900 для одноразрядных, двухразрядных и трехразрядных;
	\item запускаем генератор случайных чисел $N$ раз;
	\item определяем количество случайных чисел, попавших в каждую категорию;
	\item вычисляем значение статистики $V$;
	\item сравниваем полученное значение с теоретическими значениями в таблице, определяем к какому интервалу оно относится.
	\item делаем вывод о случайности величины, возможны три случая:
	\begin{itemize}
		\item $1\% < V_{exp} < 99\%$ - генератор удовлетворителен (однако необходимо учитывать <<подозрительные>> результаты);
		\item $V_{exp} < 1\%$ - генератор не удовлетворителен, так как разброс
		чисел слишком мал, чтобы быть случайным;
		\item $ 99\% < V_{exp}$ - генератор не удовлетворителен, так как разброс
		чисел слишком велик, чтобы быть случайным.
	\end{itemize}
\end{enumerate}




\section*{Листинги кода}
% Для листинга кода:
\lstset{ %
	language=python,                 % выбор языка для подсветки 
	basicstyle=\scriptsize\sffamily, % размер и начертание шрифта для подсветки кода
	numbers=left,               % где поставить нумерацию строк (слева\справа)
	numberstyle=\tiny,           % размер шрифта для номеров строк
	stepnumber=1,                   % размер шага между двумя номерами строк
	numbersep=5pt,                % как далеко отстоят номера строк от подсвечиваемого кода
	showspaces=false,            % показывать или нет пробелы специальными отступами
	showstringspaces=false,      % показывать или нет пробелы в строках
	showtabs=false,             % показывать или нет табуляцию в строках
	frame=single,              % рисовать рамку вокруг кода
	tabsize=2,                 % размер табуляции по умолчанию равен 2 пробелам
	captionpos=t,              % позиция заголовка вверху [t] или внизу [b] 
	breaklines=true,           % автоматически переносить строки (да\нет)
	breakatwhitespace=false, % переносить строки только если есть пробел
	escapeinside={\#*}{*)}   % если нужно добавить комментарии в коде
}



На листинге 1 приведен код класса, реализующий алгоритм линейного конгруэнтного генератора.
\begin{lstlisting}[label=CodeStand,caption=Линейный конгруэнтный генератор]
class LCGRandom:
	def __init__(self):
		self.current = 10
		self.m = 2. ** 31
		self.a = 1594525
		self.c = 1123504223
	
	def generate(self, low=0, high=100):
		self.current = (self.a * self.current + self.c) % self.m
		result = int(low + self.current % (high - low))
		return result
\end{lstlisting}


На листинге 2 приведен код функции, подсчитывающей критерий.	
\begin{lstlisting}[label=CodeStand,caption=Вычисление критерия <<Хи-квадрат>>]
def calc_chi(arr, n, start, end): 
	tab = [0 for i in range(start + end)]
	for i in range(n):
		tab[arr[i]] += 1
	s = 0
	for i in tab:
		s += i * i
	
	return s * (end - start) / n - n
\end{lstlisting}




\section*{Результаты выполнения работы}

На рисунке 1 приведены результаты работы разработанной программы.
\img{40mm}{1}{Результат запуска программы}


\section*{Вывод}
Как видно из полученных результатов, в некоторых случаях при применении табличного метода
значения оказываются <<почти подозрительными>>, однако это не критично и в среднем результаты
работы генераторов можно признать удовлетворительными. Для алгоритмического метода (алгоритма линейного конгруэнтного генератора) полученные значения статистики $V$ указывают на то, что данный метод можно признать удовлетворительным.
