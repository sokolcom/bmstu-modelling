
\section*{Цель работы}
Целью данной лабораторной работы является изучение марковских процессов: для заданной системы посчитать время, в котором находится процесс в каждом из состояний, определить устоявшиеся вероятности, а также времена стабилизации этих вероятностей.



\section*{Марковский процесс}
Марковским процесоом называют случайный процесс, протекающий в системе $S$, если он обладает следующим свойством: 
\begin{itemize}
	\item для каждого момента времени $t_0$ вероятность любого состояния системы в будущем (при $t > t_0$) зависит только от ее состояния в настоящем (при $t = t_0$);
	\item для каждого момента времени $t_0$ вероятность любого состояния не зависит от того, когда и каким образом система пришла в это состояние.
\end{itemize}
Вероятностью $i$-го состояния называется вероятность $p_i(t)$ того, что в момент $t$ система будет находиться в состоянии $S_i$. Для любого момента $t$ сумма вероятностей всех состояний равна единице.

Для решения поставленной задачи, необходимо составить систему уравнений Колмогорова по следующим принципам:
\begin{itemize}
	\item в левой части каждого из уравнений стоит производная вероятности $i$-го состояния;
	\item в правой части — сумма произведений вероятностей всех состояний (из которых идут стрелки в данное состояние), умноженная на интенсивности соответствующих потоков событий, минус суммарная интенсивность всех потоков, выводящих систему из данного состояния, умноженная на вероятность данного ($i$-го состояния).
\end{itemize}

\textbf{Пример:}

Пусть система имеет $n$ состояний. Матрица интенсивностей системы приведена в таблице 1.

\begin{table}[h!]
	\centering
	\caption{Матрица интенсивностей системы}
	\begin{tabular}{|c c c c c|} 
		\hline
		 & 0 & 1 & ... & n \\ [0.5ex] 
		\hline
		0 & 0 & $\lambda_{01}$ & ... & $\lambda_{0n}$ \\ [0.5ex] 
		\hline
		1 & $\lambda_{10}$ & 0 & ... & $\lambda_{1n}$ \\ [0.5ex] 
		\hline
		... & ... & ... & ... & ... \\ [0.5ex] 
		\hline
		n & $\lambda_{n0}$ & $\lambda_{n1}$ & ... & 0 \\ [0.5ex] 
		\hline
	\end{tabular}
\end{table}


Составим уравнения Колмогорова:
$$
\left\{\begin{matrix}
		p'_0 = -\sum_{i \in N \backslash 0} \lambda_{0i} + \sum_{i \in n \backslash 0} \Pi_{i \in n \backslash 0} \lambda_{i0}p_{i} \\
		p'_1 = -\sum_{i \in N \backslash 1} \lambda_{1i} + \sum_{i \in n \backslash 1} \Pi_{i \in n \backslash 1} \lambda_{i1}p_{i} \\
		<...>\\
		p'_n = -\sum_{i \in n \backslash n} \lambda_{ni} + \sum_{i \in n \backslash n} \Pi_{i \in n \backslash n} \lambda_{in}p_{i} \\
\end{matrix}\right.
$$



Для получения предельных вероятностей, то есть вероятностей в стационарном режиме работы при  $t \rightarrow \infty$ необходимо приравнять левые части уравнений к нулю. Таким образом получается система линейных уравнений. Для решения полученной системы необходимо добавить условие нормировки (сумма вероятностей равна 1). 

После того, как предельные вероятности будут найдены, необходимо найти время стабилизации. Для этого необходимо с интервалом  находить каждую вероятность в момент времени . Когда найденная вероятность будет равна соответствующей финальной с точностью до заданной погрешности, тогда можно завершить вычисления. На каждом шаге необходимо вычислять приращения для каждой вероятности (как функции): 

$$
dp_j = \frac{-\sum_{i \in N \backslash j} \lambda_{ji} + \sum_{i \in n \backslash j} \Pi_{i \in n \backslash j} \lambda_{ij}p_{i}}{\Delta t}
$$

Начальные значения для $dp$ задаются. Можно взять, например, 1 - для одного из состояний, а для остальных - 0, или $\frac{1}{n}$ для каждого состояния, где $n$ – количество состояний системы.





\section*{Результат выполнения задания}

\subsection*{N = 4}
На рисунке 1 приведен граф связей и интенсивности системы.
\img{130mm}{1}{Граф связей и интенсивности системы (N = 4)}


В таблице 2 приведены предельные вероятности системы.
\begin{table}[h!]
	\centering
	\caption{Предельные вероятности системы}
	\begin{tabular}{|c c c c|} 
		\hline
		p0 & p1 & p2  & p3  \\ [0.5ex] 
		\hline
		0.2 & 0.3 & 0.3 & 0.2 \\ [0.5ex] 
		\hline
	\end{tabular}
\end{table}

В таблице 3 приведены времена стабилизации при начальных условиях = $\frac{1}{n}$.
\begin{table}[h!]
	\centering
	\caption{Времена стабилизации}
	\begin{tabular}{|c c c c|} 
		\hline
		t0 & t1 & t2  & t3  \\ [0.5ex] 
		\hline
		0.869 & 0.288 & 1.038 & 0.34 \\ [0.5ex] 
		\hline
	\end{tabular}
\end{table}

В таблице 4 приведены времена стабилизации при начальных условиях = $\frac{1}{n}$.
\begin{table}[h!]
	\centering
	\caption{Времена процесса в состояниях}
	\begin{tabular}{|c c c c|} 
		\hline
		t0 & t1 & t2  & t3  \\ [0.5ex] 
		\hline
		2.0 & 3.0 & 3.0 & 2.0 \\ [0.5ex] 
		\hline
	\end{tabular}
\end{table}


На рисунках 2 и 3 приведены графики вероятностей состояний как функции времени при различных начальных условиях (точками на графиках отмечены моменты времени стабилизации).
\img{130mm}{2}{Графики вероятностей состояний как функции времен, при начальных условиях = $\frac{1}{n}$}
\img{130mm}{3}{Графики вероятностей состояний как функции времен, при начальных условиях = 1 для одного состояния и 0 для остальных}




\subsection*{N = 5}
На рисунке 4 приведен граф связей и интенсивности системы.
\img{130mm}{4}{Граф связей и интенсивности системы (N = 5)}


В таблице 5 приведены предельные вероятности системы.
\begin{table}[h!]
	\centering
	\caption{Предельные вероятности системы}
	\begin{tabular}{|c c c c c|} 
		\hline
		p0 & p1 & p2  & p3 & p4  \\ [0.5ex] 
		\hline
		0.311 & 0.232 & 0.208 & 0.149 & 0.1 \\ [0.5ex] 
		\hline
	\end{tabular}
\end{table}

В таблице 6 приведены времена стабилизации при начальных условиях = $\frac{1}{n}$.
\begin{table}[h!]
	\centering
	\caption{Времена стабилизации}
	\begin{tabular}{|c c c c c|} 
		\hline
		t0 & t1 & t2  & t3 & t4 \\ [0.5ex] 
		\hline
		0.953 & 5.663 & 6.189 & 0.56 & 7.675 \\ [0.5ex] 
		\hline
	\end{tabular}
\end{table}

В таблице 7 приведены времена стабилизации при начальных условиях = $\frac{1}{n}$.
\begin{table}[h!]
	\centering
	\caption{Времена процесса в состояниях}
	\begin{tabular}{|c c c c c|} 
		\hline
		t0 & t1 & t2  & t3 & t4 \\ [0.5ex] 
		\hline
		3.114 & 2.318 & 2.076 & 1.495 & 0.997 \\ [0.5ex] 
		\hline
	\end{tabular}
\end{table}

На рисунках 5 и 6 приведены графики вероятностей состояний как функции времени при различных начальных условиях (точками на графиках отмечены моменты времени стабилизации).
\img{130mm}{5}{Графики вероятностей состояний как функции времен, при начальных условиях = $\frac{1}{n}$}
\img{130mm}{6}{Графики вероятностей состояний как функции времен, при начальных условиях = 1 для одного состояния и 0 для остальных}




\subsection*{N = 6}
На рисунке 7 приведен граф связей и интенсивности системы.
\img{130mm}{7}{Граф связей и интенсивности системы (N = 6)}

В таблице 8 приведены предельные вероятности системы.
\begin{table}[h!]
	\centering
	\caption{Предельные вероятности системы}
	\begin{tabular}{|c c c c c c|} 
		\hline
		p0 & p1 & p2  & p3 & p4 & p5  \\ [0.5ex] 
		\hline
		0.175 & 0.105 & 0.32 & 0.143 & 0.079 & 0.178 \\ [0.5ex] 
		\hline
	\end{tabular}
\end{table}

В таблице 9 приведены времена стабилизации при начальных условиях = $\frac{1}{n}$.
\begin{table}[h!]
	\centering
	\caption{Времена стабилизации}
	\begin{tabular}{|c c c c c c|} 
		\hline
		t0 & t1 & t2  & t3 & t4 & t5 \\ [0.5ex] 
		\hline
		0.267 & 2.872 & 2.697 & 1.129 & 2.546 & 0.426 \\ [0.5ex] 
		\hline
	\end{tabular}
\end{table}

В таблице 10 приведены времена стабилизации при начальных условиях = $\frac{1}{n}$.
\begin{table}[h!]
	\centering
	\caption{Времена процесса в состояниях}
	\begin{tabular}{|c c c c c c|} 
	\hline
	t0 & t1 & t2  & t3 & t4 & t5 \\ [0.5ex] 
	\hline
	1.749 & 1.055 & 3.198 & 1.432 & 0.789 & 1.777 \\ [0.5ex] 
	\hline
\end{tabular}
\end{table}

На рисунках 8 и 9 приведены графики вероятностей состояний как функции времени при различных начальных условиях (точками на графиках отмечены моменты времени стабилизации).
\img{130mm}{8}{Графики вероятностей состояний как функции времен, при начальных условиях = $\frac{1}{n}$}
\img{130mm}{9}{Графики вероятностей состояний как функции времен, при начальных условиях = 1 для одного состояния и 0 для остальных}
